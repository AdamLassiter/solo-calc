Over the course of the project, a set of interacting systems have been built.
A representation of a calculus expression may be built from a textual input.
From here, it may be converted to a normal form, which itself can be trivially converted to a diagram.
\textit{With further work, I would like to have seen a connect between the REPL calculus interface and the diagram REST server.}
This normal form may then be reduced to a simpler expression, with functions for testing for $\alpha$-equivalence.
Diagrams may be constructed, albeit only through hard-coding them into the application.
These diagrams can be reduced and converted to an abstract representation through JSON.
Each of these diagrams may then be visualised, manipulated and examined through each reduction step.
These visualisations of the diagram objects are believed to be clear and allow for a more intuitive understanding of the subject of process calculi.\\

Through this project's development, \textit{I believe that the diagram objects serve as a superior representation of the calculus.}
The mathematical purity to Solo Diagrams is something to be admired.
This, combined with the relative lack of further research, leaves a strong case for deeper investigation into Solo Diagrams.\\

The shortfalls of the project can mostly be attributed to implementation difficulties in the technicalities of the Solo Calculus.
However, there could be much improvement on the visualiser, for better differentiating cases of ambiguity and for allowing richer interaction on the user end.
\textit{Through further work, I would like to see the ability to build, destroy and edit diagrams from within the visualisation program.}\\

The specifics of finding general solutions to usually very human-solvable problems such as $\alpha$-equivalence and construction of suitable $\sigma$s in reduction steps has given new perspective on the implied vs in-practice hardness of said problems.
Contrary to the comment made in~\ref{sssec:diagram-implementation-analysis} as to reduction of diagrams being done through conversion to calculus expressions, \textit{I believe it would be easier to approach the problem from the opposite direction.}
\textit{Implementation of the Solo Calculus would be eased through converting expressions to diagrams and reducing through their set of rules instead.}
However, calculus expressions remain the obvious choice as an interface in situations concerning textual input and output.\\

Future research and development towards the subject of Proof Nets could prove influential and extensive.
The reader is referred to later sections in the work of~\cite{solo-diagrams} and of~\cite{acyclic-solos}.
