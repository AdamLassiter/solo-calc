\subsection{Requirements}
    
    The following requirements seek to provide an estimate for the progress path the project will take.
    The priorities of each requirement are \textit{must > should > may > might}.
    Where these changed over time and where requirements were not met has been noted.
   
    \subsubsection{Solo Calculus}
        The first part of the system will be focused on text-based calculus expressions only.
        This will form a basis upon which to implement other features.
        This system:
        \begin{itemize}
            \item \textit{Must} support input and output of text representing expressions.
            \item \textit{Must} be able to perform reductions on inputted expressions and return these reductions.
            \item \textit{Should} provide this functionality though a accessible and useable user interface.
            \item \textit{Might} be implemented to evaluate expressions concurrently.
                \textit{This was not achieved over the course of the project.}
        \end{itemize}

    \subsubsection{Solo Diagrams}
        This will be the first extension of the system, allowing interoperability with diagrams.
        This system:
        \begin{itemize}
            \item \textit{Must} support input and output of diagrams.
                \textit{While this was achieved during the project, it was only in part.
                The diagrams used remained hard-coded in the application.
                It was hoped that this could have been achieved through inputting a calculus expression using the above, converting to a diagram and using it.}
            \item \textit{Should} provide a conversion between diagrams and expressions.
                \textit{Again, see above as to how this could have been achieved.}
            \item \textit{Should} provide the same reductions as are provided to calculus expressions.
            \item \textit{May} be able to produce step-by-step reductions.
            \item \textit{Might} be implemented to evaluate diagrams concurrently.
                \textit{Again, this was not achieved over the course of the project.
                The construction of diagrams allows for easily dividing into independent subdiagrams, which may be reduced with no interaction with one another.
                This would allow for a simple method to partition the dataset across multiple processes.}
        \end{itemize}

    \subsubsection{Diagram Visualisation}
        The second extension will build on the previous Solo Diagrams and provide a graphical interface for input/output/manipulation of diagrams.
        This should provide a clear visual real-time representation of how the calculus works.
        This system:
        \begin{itemize}
            \item \textit{Must} provide a visualisation of diagrams through a graphical user interface.
            \item \textit{Should} support input/output of a diagram from/to an expression.
                \textit{Again, see above as to how this could have been achieved.}
            \item \textit{Should} support manipulation of a diagram within the GUI.
            \item \textit{Should} be both intuitive and feature-complete to provide an obvious improvement over the text-based input of expressions as described above.
            \item \textit{May} be able to produce step-by-step reductions.
        \end{itemize}
