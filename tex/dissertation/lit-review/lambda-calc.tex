\subsection{\texorpdfstring{$\lambda$-c}{Lambda C}alculus}\label{ssec:lambda-calculus}

    Developed by Alonzo Church in the 1930s, the $\lambda$-calculus was the first such computational calculus and describes a mathematical representation of a computable function.
    When it was first designed, it was not expected to be as relevant to the newly-emerging field of theoretical Computer Science but instead Discrete Mathematics and Category Theory, the $\lambda$-calculus in fact forms a universal model of computation and can contain an encoding of any single-taped Turing machine.
    It has since become a popular formal system within which to study properties of computation.


    \begin{definition}{Syntax\\}
        \label{lambda-calculus-syntax}
        The $\lambda$-calculus, as defined by Church but here explained by~\cite{lambda-calculus}\footnotemark consists of an expression $M$ built from the following terms:
        \footnotetext{While Church's original paper is still available, the source cited is found to be more relevant due to research in the subject area since the original paper's publication in the 1930s as well as a narrower scope than Church initially provides.}
        \begin{center}
            \begin{tabular}{ l l l }
                $M  \quad \defeq$       & $a$               & (variable) \\
                                        & $\lambda x . M$   & (abstraction) \\
                                        & $M N$             & (application)
            \end{tabular}
        \end{center}
    \end{definition}
    From this, any computable function can be constructed and computation is achieved through a series of operations on the expression.


    \begin{definition}{$\alpha$-substitution\\}
        \label{lambda-alpha-relation}
        Unbound variables within an expression may be substituted for any given value.
        This is formally expressed as:
        \begin{align*}
            x[y \defeq P] \defeq &
                \begin{cases}
                    P & \text{if } x = y \\
                    x & \text{otherwise}
                \end{cases} \\
            (\lambda x . M)[y \defeq P] \defeq &
                \begin{cases}
                    \lambda x . M & \text{if } x = y \\
                    \lambda x . (M[y \defeq P]) & \text{if } x \neq y \text{ and } x \notin FV(P)~\footnotemark
                \end{cases}
        \end{align*}\footnotetext{Here, $FV(P)$ is the set of all variables $x$ such that $x$ is free (unbound) in $P$.}
    \end{definition}
    This operation may be thought of as variable renaming as long as both old and new names are free in the expression in which they were substituted.

    \begin{corollary*}{$\alpha$-equivalence\\}
        The above definition of $\alpha$-substitution may be extended to give an equivalence relation on expressions, $\alpha$-equivalence, defined as:
        \begin{align*}
            y \notin FV(M) & \implies \lambda x . M \equiv_\alpha \lambda y \, (M[x \defeq y]) \\
            M \equiv_\alpha M' & \implies
                \begin{cases}
                    M \, P \equiv_\alpha M' \, P \\
                    P \, M \equiv_\alpha P \, M' \\
                    \lambda x . M \equiv_\alpha \lambda x. M'
                \end{cases}
        \end{align*}
    \end{corollary*}


    \begin{definition}{$\beta$-reduction\\}
        An expression may be simplified by applying one term to another through substitution of a term for a bound variable.
        This is formally expressed as:
        \begin{align*}
            (\lambda x . P) \, Q   & \rightarrow_\beta P[x \defeq Q] \\
            M \rightarrow_\beta M' & \implies
                \begin{cases}
                    P \, M \rightarrow_\beta P M' \\
                    M \, P \rightarrow_\beta M' P \\
                    \lambda x . M \rightarrow_\beta \lambda x . M'
                \end{cases}
        \end{align*}
    \end{definition}
    Often $\beta$-reduction requires several steps at once and as such these multiple $\beta$-reduction steps are abbreviated to $\rightarrow_\beta^*$.

    \begin{corollary*}{$\beta$-equivalence\\}
        The above definition of $\beta$-reduction may be extended to give an equivalence relation on expressions, $beta$-equivalence, defined as:
        \begin{equation*}
            M \rightarrow_\beta^* P \text{ and } N \rightarrow_\beta^* P \implies M \equiv_\beta N
        \end{equation*}
    \end{corollary*}
    

    \begin{example*}
        The above corollaries can be seen to have desirable properties when examining whether two expressions describe equivalent computation.
        \begin{align*}
            \lambda a . x \, a [x \defeq y] & \equiv \lambda a . y \, a \\
            \\
            \lambda x . x \, y & \equiv_\alpha \lambda z . z \, y \\
            \lambda x . x \, y & \not\equiv_\alpha \lambda y . y \, y \\
            \\
            (\lambda a . x \, a) \, y & \rightarrow_\beta x \, a [a \defeq y] \equiv x \, y
        \end{align*}
    \end{example*}


    As computational calculus shares many parallels with modern functional programming, the following are encodings of some common functional concepts within the $\lambda$-calculus.

    \begin{definition}{List\\}
        Within the $\lambda$-calculus, lists may be encoded through the use of an arbitrary ${cons}$ function that takes a head element and a tail list and of a ${null}$ function that signifies the end of a list.
        The list is then constructed as a singly-linked list might be constructed in other languages:
        \begin{equation*}
            [x_1, \ldots, x_n] \defeq \lambda c . \lambda n . (c \, x_1 \, (\ldots (c \, x_n \, n) \ldots) )
        \end{equation*}
    \end{definition}


    \begin{definition}{Map\\}
        The $map$ function takes two arguments --- a function $F$ that itself takes one argument and a list of suitable arguments $[x_1, \ldots, x_n]$ to this function.
        The output is then a list of the output of F when applied to each $x_1 \ldots x_n$.
        \begin{equation*}
            {map} \defeq \lambda f . \lambda l . (\lambda c . (l \, (\lambda x . c \, (f \, x))))
        \end{equation*}
    \end{definition}


    \begin{example*}
        As follows is an example of the reductions on the ${map}$ function for a list of length $n \defeq 3$:
        \begin{align*}
            {map} \, F \, [x_1, x_2, x_3]
                  & \equiv_{\alpha} \lambda f . \lambda l . (\lambda c . (l \, (\lambda x . c \, (f \, x)))) \, F \, \lambda c . \lambda n . (c \, x_1 \, (c \, x_2 \, (c \, x_3 \, n))) \\
                  & \rightarrow_\beta^* \lambda c . (\lambda c . \lambda n . (c \, x_1 \, (c \, x_2 \, (c \, x_3 \, n))) \, (\lambda x . c \, (F \, x))) \\
                  & \rightarrow_\beta^* \lambda c . (\lambda n . ((\lambda x . c \, (F \, x)) \, x_1 \, ((\lambda x . c \, (F \, x)) \, x_2 \, ((\lambda x . c \, (F \, x)) \, x_3 \, n)))) \\
                  & \rightarrow_\beta^* \lambda c . (\lambda n . (c \, (F \, x_1) \, (c \, (F \, x_2) \, (c \, (F \, x_3) \, n)))) \\
                  & \equiv_{\alpha} [F \, x_1, F \, x_2, F \, x_3]
        \end{align*}
    \end{example*}

    
    \begin{remarks}
        While the $\lambda$-calculus has been successful and been studied by various areas of academia outside of Computer Science, it is limited in modern-day application by its fundamentally `single-process' model and struggles to describe multiple systems working and communicating together. 
        While certain additional properties --- specifically the $Y$-combinator and simply-typed $\lambda$-calculus --- are not mentioned here, the calculus is defined from a few simple rules.
        This simplicity allows implementations of $\lambda$-calculus interpreters to be relatively painless.\footnote{There exists an example of such an interpreter, available online at the time of writing, at \url{http://www.cburch.com/lambda/}}
        \textit{In my opinion, the simplicity and expressiveness of the $\lambda$-calculus should be the standard to which other computational calculi are held.}\\

        The ${map}$ example is particularly relevant to study within concurrent calculi as multiple large-scale systems follow a MapReduce programming model, as described by~\cite{mapreduce}, which utilises massive parallelism of large detacenters.
        The model requires a ${map}$ function that applies a function to a key and set of values, similar to that described above, and a ${reduce}$ function that collects all values with matching keys.
        This has found to be useful for modelling many real-world tasks for performance reasons, but concurrent calculi may provide a simple case for study and understanding for any of these tasks which scales as necessary.
    \end{remarks}
