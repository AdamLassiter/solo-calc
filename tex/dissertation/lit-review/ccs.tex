\subsection{Calculus of Communicating Systems}

    The Calculus of Communicating Systems (CCS), as described by~\cite{ccs-syntax}, was one of the earlier\footnotemark{} process calculi.
    \footnotetext{The reader is referred to `Communicating Sequential Processes' by~\cite{csp}, which may be thought of as the first such concurrent process calculus.
                  It is excluded here as it bears more similarities with a traditional programming language and is therefore less relevant.}
    It was designed in the same vein as Church's $\lambda$-calculus, but with a focus on modelling concurrent systems.
    Among the many differences, most notable are the ability for concurrency and synchronisation through waiting on input/output through names.
    For reasons discussed in~\ref{subsec:pi-calculus}, it did not become as mainstream as the $\lambda$-calculus but did serve as an important basis for study in the subject distinct from Church's single-process model.


    \begin{definition}{(Syntax)\\}
        Within CCS, a process $P$ is defined as:
        \begin{center}
            \begin{tabular}{ l l l }
                $P \quad \defeq$    & $nil$ or $0$              & (inaction process) \\
                                    & $x$                       & (variable) \\
                                    & $\tau$                    & (silent action) \\
                                    & $a x_1 \ldots x_n$        & (action on $x_1 \ldots x_n$)~\footnotemark\\
                                    & $P \, | \, Q$             & (composition) \\
                                    & $P + Q$                   & (choice / summation)\\
                                    & $P \backslash a$          & (restriction) \\
                                    & $P[b / a]$                & (relabelling, a $\defeq$ b)~\footnotemark\\
                                    & $x(x_1 \ldots x_n)$       & (identifier) \\
                                    & if $x$ then $P$ else $Q$  & (conditional)
            \end{tabular}
        \end{center}
        \addtocounter{footnote}{-2}\footnotetext{These actions come in pairs $a$ and $\bar{a}$ representing input and output respectively.}
        \stepcounter{footnote}\footnotetext{There may be multiple relabellings at once, so this is often written $p[S]$ where the function $S$ has $dom(S) = \{a\}\, ran(S) = \{b\}$}
    \end{definition}
    When comparing to the $\lambda$-calculus from Definition~\ref{lambda-calculus-syntax}, certain parallels can be seen.
    Notable additions are the action operator and the composition operators (parallel composition and choice). \\


    Due to the increased amount of syntax, CCS has many more rules and semantics for computation than the $\lambda$-calculus.
    Here $\xrightarrow{a}$ describes reduction through taking an arbitrary action $a$.

    \begin{definition}{(Action Semantics)\\}
        Given two processes waiting on input/output, eventually an input/output action will happen.
        This gives the reductions as follows:
        \begin{align}
            a x_1 \ldots x_n . P        & \xrightarrow{a v_1 \ldots v_n} P[v_1 / x_1 \ldots v_n / x_n] \\
            \bar{a} v_1 \ldots v_n . P  & \xrightarrow{\bar{a} v_1 \ldots v_n} P \\
            \tau . P                    & \xrightarrow{\tau} P
        \end{align}
    \end{definition}


    \begin{definition}{(Composition Semantics)\\}
        For such an action $a$ taking place, the following reductions can be made:
        \begin{align}
            P \xrightarrow{a} P' \implies &
            \begin{cases}
                P + Q \xrightarrow{a} P' \\
                P \, | \, Q \xrightarrow{a} P' \\
            \end{cases} \\
            \begin{rcases}
                P \xrightarrow{a} P' \\
                Q \xrightarrow{\bar{a}} Q' \\
            \end{rcases} \implies &
            P \, | \, Q \xrightarrow{\tau} P' \, | \, Q'
        \end{align}
    \end{definition}
    The choice here of whether to follow the left or right side is non-deterministic.
    This leads to what is described as the Tea/Coffee Problem.


    \begin{example*}{(Tea/Coffee Problem)\\}
        Suppose there is a machine that, when given a coin, will dispense either tea or coffee.
        A user comes to insert a coin to get some tea.
        This system could be described as:
        \begin{equation*}
            coin \, . \, \overline{tea} \, . \, 0 \, + \, coin \, . \, \overline{coffee} \, . \, 0 \quad | \quad \overline{coin} \, . \, tea \, . \, 0
        \end{equation*}
        But this would be incorrect.
        After the $\xrightarrow{coin}$ action, a choice would need to be made as to whether prepare to output tea or to output coffee.
        Only in the case that it is decided to output tea does the system halt successfully.
        Otherwise, it is left in the state:
        \begin{equation*}
            \overline{coffee} \, . \, 0 \quad | \quad tea \, . \, 0
        \end{equation*}
        
        The problem would instead be successfully encoded as:
        \begin{equation*}
            coin \, . \, (\overline{tea} \, . \, 0 \, + \, \overline{coffee} \, . \, 0) \quad | \quad \overline{coin} \, . \, tea \, . \, 0
        \end{equation*}
        It is important here to note that the \textit{trace} of both programs (the set of inputs that produce accepted outputs) is identical, but the two are not \textit{bisimilar} (they are equivalent to the actions that can be taken at any step).
        These concepts will be examined further later.
    \end{example*}


    \begin{remark*}{(A Note on Traces and Bisimulation)\\}
        The idea of both \textit{trace} and \textit{bisimulation} are attempts to define a system for analysing \textit{behavioural equivalence}.
        That is, given two programs that are written very different but behave similarly, is their behaviour exactly equivalent. \\

        The trace comes from automata theory and is exactly equivalent to \textit{language equivalence} --- given a set of inputs, do both programs accept and reject (or produce the same output) for all the same inputs.
        Trace equivalence is considered the weakest equivalence for two systems. \\

        Bisimulation however holds roots from a more mathematical standpoint --- does there exist a bijection between behaviours of each system \textit{at any given step}.
        More specifically, two processes $P, Q$ are bisimilar (written $P \, \mathcal{R} \, Q$) if:
        \begin{equation*}
            \begin{rcases}
                P \, \mathcal{R} \, Q \\
                P \xrightarrow{\alpha} P' \\
            \end{rcases} \implies \exists Q' \, :
            \begin{cases}
                Q \xrightarrow{\alpha} Q' \\
                P' \, \mathcal{R} \, Q' \\
            \end{cases}
        \end{equation*}
        This is a much stronger property and is one of the strongest that can be shown except for $\alpha$-equivalence (identical up to name-substitution). \\
        
        From the Tea/Coffee Problem, it can be seen that the two systems are trace-equivalent.
        However, they are not behaviourally-equivalent.
        Trace-equivalence is less useful once the restriction on deterministic processes or on all input being provided at once is removed.
    \end{remark*}


    \begin{definition}{(Restriction Semantics and Relabelling)\\}
        Restrictions and relabellings hold the property:
        \begin{align}
            P \xrightarrow{a x} P' \implies
            \begin{cases}
                P \backslash b \xrightarrow{a x} P ' \backslash b \quad \text{if } a \notin \{b, \bar{b}\}\\
                P[S] \xrightarrow{S(a) x} P'[S] \\
            \end{cases}
        \end{align}
    \end{definition}
    That is, a process is equivalent under renaming if any actions on that process are also renamed.


    \begin{definition}{(Identifier Semantics)\\}
        Suppose a behaviour identifier $b$ is defined (possibly recursively) as $b(x_1 \ldots x_n) \impliedby P$, and that for the process $P$, $FV(P) \subseteq \{x_1 \ldots x_n\}$.
        Then processes may be reduced as follows:
        \begin{align}
            P[v_1 / x_1 \ldots v_n / x_n] \xrightarrow{a x} P' \implies
            b(v_1 \ldots v_n) \xrightarrow{a x} P'
        \end{align}
    \end{definition}
    The identifier operation can be seen as similar to abstractions ($\lambda x . M$) in the $\lambda$-calculus.


    \begin{remarks}
        CCS excels in providing a powerful language for describing high-level concurrent systems.
        Note the limit on only inputting and outputting variables and expressions, as well as the asynchronous nature of inter-process communication.
        However it struggles to describe the low-level atomic actions.
        An encoding of, say, a list is difficult as the language revolves around systems communicating with one another through input/output synchronisation. \\

        The Tea/Coffee Problem should be kept in mind for the following calculi, particularly for the Solo calculus.
    \end{remarks}
