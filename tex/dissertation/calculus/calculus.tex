\subsection{Calculus Implementation}\label{ssec:calculus-implementation}
    
    \subsubsection{Analysis}\label{sssec:calculus-analysis}
        The implementation of reduction semantics and calculus objects was achieved through converting all expressions to a canonical normal form.
        First, we must gather some possible structural congruences.

        \begin{lemma}{Structural Equivalence\\}
            For any agents $P, Q, \ldots$, the following structural equivalences hold.
            \begin{align*}
                (x)(y) P                & \equiv (x \, y) P \\
                P \, | \, (Q \, | \, R) & \equiv P \, | \, Q \, | \, R \\
                P \, | \, (x) Q         & \equiv (z)(P \, | \, Q\{z / x\}) \quad \text{where} \quad z \notin names(P) \cup names(Q) \\
                !(x)(P \, | \, !Q)      & \equiv (z)(!(x)(P \, | \, z \, \tilde{x}) \, | \, !(\tilde{w})(\bar{z} \, \tilde{w} \, | \, Q\{\tilde{w}/\tilde{x}\})) \quad \text{where} \quad \tilde{x} \defeq fn(P)
            \end{align*}
            Using these congruences, we may reorder terms of expressions through $\alpha$-equivalence only.
        \end{lemma}

        \begin{definition}{Normal Form\\}
            An agent $P$ is of normal form \textit{iff}
            \begin{align*}
                                    P \quad &\equiv \quad (\tilde{x})(\prod_{i}{y_i \, \tilde{z_i}} \; | \; \prod_{j}{! \, Q_j}) \\
                \text{where} \quad Q_j \quad &\equiv \quad (\tilde{x})(\prod_{k}{y_k \, \tilde{z_k}})
            \end{align*}
            That is, P is written as a 3-tuple of a scope, a composition of solos and a composition of replicators, where each replicator is written as a 2-tuple of a scope and a composition of solos only.
        \end{definition}

        \begin{lemma}{~\\}
            For any agent P, $\exists \, Q \equiv P$ such that Q is in normal form.
            \begin{proof}
                The proof is trivial by applying recursively the equivalences above.
                See below for an algorithm for such a construction.
            \end{proof}
        \end{lemma}

        \begin{breakablealgorithm}
            \caption{Construction of Normal Forms}
            \begin{algorithmic}[1]
                \Require{Agent $P$ a Normal Form, $Q$ non-Normal Form}
                \Ensure{Agent $P'$, Normal Form of $P$ and $Q$}
                \Function{Normalise}{$P$}
                    \If{$Q$ a Scope}
                        \State$(\tilde{bn})Q' \defeq Q$
                        \State$collisions \defeq \tilde{bn} \cap names(P)$
                        \If{$collisions \neq \emptyset$}
                            \State$\alpha : collisions \rightarrow fresh\;names$
                            \State\Return\Call{Normalise}{$P$, $\alpha(Q)$}
                        \Else
                            \State\Return$(\tilde{bn})$\Call{Normalise}{$P$, $Q'$}
                        \EndIf\\

                    \ElsIf{$Q$ a Composition}
                        \State$Q_1 \;|\; \ldots \;|\; Q_n \defeq Q$
                        \State\Return\Call{Reduce}{\texttt{Normalise}, $P$, $Q_1 \ldots Q_n$}\\

                    \ElsIf{$Q$ a Replication}
                        \State$Q' \defeq$~\Call{Normalise}{$Q''$} where $!Q'' \defeq Q$
                        \If{$replicators(Q') \neq \emptyset$}
                            \State$\bar{Q} \defeq$~\Call{Flatten}{Q}
                            \State\Return\Call{Normalise}{$P$, $\bar{Q}$}
                        \Else
                            \State$collisions \defeq bn(P) \cap bn(Q')$
                            \State$\alpha : collisions \rightarrow fresh\;names$
                            \State\Return$P \; | \; !\,\alpha(Q')$
                        \EndIf\\

                    \ElsIf{$Q$ a Solo}
                        \State$(\tilde{bn})P' \defeq P$
                        \State$collisions \defeq \tilde{bn} \cap names(Q)$
                        \If{$collisions \neq \emptyset$}
                            \State$\alpha : collisions \rightarrow fresh\;names$
                            \State\Return\Call{Normalise}{$\alpha(P)$, $Q$}
                        \Else
                            \State\Return$(\tilde{bn})(P' \; | \; Q)$
                        \EndIf
                    \EndIf
                \EndFunction
            \end{algorithmic}
        \end{breakablealgorithm}
        Normalisation is then performed on an agent $P$ by $\texttt{Normalise}(P) \defeq \texttt{Normalise}(0, P)$
        It will be seen later that this is exactly equivalent to how Solo diagrams are represented.
        From here onwards, it is assumed that any agent has already been converted to normal form where appropriate.\\

        The algorithm for reductions is a simple search problem for two solos of matching subject and arity (number of names) and opposite parity (input vs. output).
        \begin{breakablealgorithm}
            \caption{Reduction of Solos}
            \begin{algorithmic}[1]
                \Require{Agent $P$}
                \Ensure{Agent $P'$, a reduction of $P$}
                \Function{Reduce}{$P$}
                    \For{$i \in inputs(P)$} \Comment{$inputs(P) \equiv \{x \in solos(P) \;|\; x \text{ an input}\}$}
                        \For{$o \in outputs(P)$} \Comment{$outputs(P) \equiv \{x \in solos(P) \;|\; x \text{ an output}\}$}
                            \If{$i$ agrees with $o$} \Comment{`agrees' $\equiv$ matching subject and arity}
                                \State$\sigma :=$ \Call{Fuse}{$i$, $o$, $bn(P)$}
                                \If{$\sigma \neq \textit{none}$}
                                    \State\Return$\sigma(P - \{i, o\})$
                                \EndIf
                            \EndIf
                        \EndFor
                    \EndFor
                    \State\Return$P$
                \EndFunction
            \end{algorithmic}
        \end{breakablealgorithm}

        The next step is to find a suitable $\sigma$, should one exist. This is a renaming of the names of $P$ and must satisfy:
        \begin{align*}
            \sigma : \; & bn(P) \rightarrow names(P) \\
            \emptyset = \; & domain(\sigma) \cap range(\sigma) \\
            \sigma(x) = \; & y \implies \text{$x$ and $y$ have been fused}
        \end{align*}

        This is found through converting the list of pairs of object names to edges of a graph.
        For each disconnected subgraph, or partition, the span of names forms a set which must all be fused into one another.
        Subsequently, there must be at most one free name in this set as two or more would require renaming one free name to another, which is not allowed.
        If no free name exists, a bound name is chosen as the `free name' at random.
        $\sigma$ is then constructed by $\sigma(bn) := fn$.

        \begin{breakablealgorithm}
            \caption{Fusion of Solos}
            \begin{algorithmic}[1]
                \Require{$i_1 \ldots i_n \text{ objects of solo } i, \quad o_1 \ldots o_n \text{ objects of solo } o, \quad bn \; \text{set of bound names}$}
                \Ensure{$\sigma : bn \rightarrow names(i) \cup names(o)$ or \textit{none}}
                \Function{Fuse}{$i_1 \ldots i_n$, $o_1 \ldots o_n$, $bn$}
                    \State$\texttt{Graph} \; g := \{(i_j, o_j) \; | \; 1 \leq j \leq n\}$
                    \State$\texttt{Map} \; \sigma := id$
                    \ForEach{$\texttt{Graph} \; \bar{g} \in partitions(g)$}
                        \State$isect := nodes(\bar{g}) - bn$
                        \If{$|isect| = 0$}
                            \State$fn := x \in nodes(\bar{g})$
                        \ElsIf{$|isect| = 1$}
                            \State$fn := x \in isect$
                        \Else
                            \State\Return\textit{none}
                        \EndIf
                        \ForEach{$name \in nodes(\bar{g}) - \{fn\}$}
                            \State$\sigma(name) := fn$
                        \EndFor
                    \EndFor
                    \State\Return$\sigma$
                \EndFunction
            \end{algorithmic}
        \end{breakablealgorithm}

        This forms a complete implementation of non-replicating parts of the calculus.
        To avoid the implementation problems with replicators mentioned in~\cite{solo-diagrams}, it is necessary to only perform expansions on replicators that can be reduced.
        While the method described in the aforementioned paper would be suitable, it is enough to simply search for replicators which may be reduced, but only expand them rather than perform a complete reduction.
        This eases some implementation details and minimises code duplication while remaining correct.
        The expansion will be reduced on the next pass of the \texttt{Reduce} function.
        Below shows an extension of the \texttt{Reduce} function for a replicator-replicator fusion (reduction of the form $!P \;|\; !Q$).

        \begin{breakablealgorithm}
            \caption{Reduction of Replicators}
            \begin{algorithmic}[1]
                \Require{Agent $P$}
                \Ensure{Agent $P'$, a reduction of $P$}
                \Function{Reduce}{$P$}
                    \For{$i \in inputs(P)$}
                        \State$\ldots$
                    \EndFor\\

                    \For{$i \in \bigcup_{r \in reps(P)} inputs(r)$} \Comment{$reps(P) \equiv \{x \in P \;|\; x \text{ a replicator}\}$}
                        \For{$o \in \bigcup_{r \in reps(P)} outputs(r)$}
                            \If{$i$ agrees with $o$}
                                \State$\sigma :=$ \Call{Fuse}{$i$, $o$, $bn(P) \cup bn(r_i) \cup bn(r_o)$} \Comment{$r_x \equiv r \in reps(P) \;|\; x \in solos(r)$}
                                \If{$\sigma \neq \textit{none}$}
                                    \State\Return$P \cup \{expand(r_i), \; expand(r_o)\}$ \Comment{$expand(!R) \equiv R$}
                                \EndIf
                            \EndIf
                        \EndFor
                    \EndFor
                    \State\Return$P$
                \EndFunction
            \end{algorithmic}
        \end{breakablealgorithm}

        The third case of solo-replicator fusion (reduction of the form $P \;|\; !Q$) is then trivial.


    \subsubsection{Testing and Correctness}
        Testing of this section was integrated with testing of the REPL by using the helper functions it provided for constructing agents.
        An in-depth description can be found in~\ref{sssec:repl-testing}. \\
        
        Early tests could be tested by string equalities and this was sufficient.
        However, more complicated agents led to non-deterministic results and non-trivial equalities.
        The first example of this occured while comparing the following agents:
        \begin{align*}
            (x) p \, x \quad\equiv\quad (y) p \, y \quad\neq\quad (x) p \, y
        \end{align*}
        Before testing could be achieved in an automated manner, an algorithm for testing for $\alpha$-equivalence was first required.\\

        It will be seen later that conversion of calculus expressions into diagrams intuitively shows $\alpha$-equivalence is reducible to graph isomorphism, so is expected to be of poor performance.
        \begin{breakablealgorithm}
            \caption{$\alpha$-Equivalence of Agents}
            \begin{algorithmic}[1]
                \Require{Agents $P, Q$}
                \Ensure{Map $\alpha : names(P) \rightarrow names(Q) \textit{ s.t. } \alpha(P) \equiv Q$ or \textit{none}}
                \Function{$\alpha$-Equivalence}{$P$, $Q$}
                    \If{$P \equiv Q$}
                        \State\Return$id$
                    \EndIf
                    \For{$comb_{scope} \in combinations(scope(P), scope(Q))$}
                        \State$\alpha_{P, Q} : x \mapsto y \; \forall \; (x, y) \in comb_{scope}$
                        \For{$comb_{replicator} \in combinations(replicators(P), replicators(Q))$}
                            \ForEach{$(r_P, r_Q) \in comb_{replicator}$}
                                \State$\alpha_{r_P, r_Q} \defeq$ \Call{$\alpha$-Equivalence}{$r_P, r_Q$}
                            \EndFor
                            \If{$\textit{none} \in \{\alpha_{r_P, r_Q} | (r_P, r_Q) \in comb_{replicator}\}$}
                                \State\textbf{continue for}
                            \ElsIf{$\exists \; \alpha_1, \alpha_2 \in \{\alpha_{r_P, r_Q}\} \textit{ s.t. } \alpha_1 \text{ disagrees with } \alpha_2$}
                                \State\textbf{continue for}
                            \Else
                                \State$\alpha \defeq \alpha_{P, Q} \cup \bigcup_{\alpha \in \{\alpha_{r_P, r_Q}\}} \alpha$
                            \EndIf
                            \If{$\alpha(P) \equiv \alpha(Q)$}
                                \State\Return$\alpha$
                            \EndIf
                        \EndFor
                        \If{$\alpha(P) \equiv \alpha(Q)$}
                            \State\Return$\alpha$
                        \EndIf
                    \EndFor
                    \State\Return\textit{none}
                \EndFunction
            \end{algorithmic}
        \end{breakablealgorithm}


    \subsubsection{Discussion}
        The initial implementation stayed closer to the core concepts of the calculus.
        There existed separate classes of objects for each of Scopes, Compositions, Replications, Matches and Solos.
        While such an approach is perhaps more intuitive, problems arise due to conflicts between how the data structure naturally appears to be tree-like where properties of agents are functions of themselves and their children only and how whether or not a name is bound or free is a property dependent upon an agent or its parent, grandparent etc.

        \begin{example*}
            Consider the following equivalent expressions.
            \begin{align*}
                P \quad & \defeq \quad (y)(a \, x \,|\, (\bar{a} \, y \,|\, p \, y)) \\
                Q \quad & \defeq \quad (a \, x \,|\, (y)(\bar{a} \, y \,|\, p \, y))
            \end{align*}
            Both $P$ and $Q$ have the property that $y$ is a bound name and both should reduce to the term $p \, x$.
            In both cases, $y$ and $x$ are fused into the free name $x$. The scope of $y$ disappears and the remaining `print' term displays the result of the fusion.
            Consider now the (inequivalent) expression.
            \begin{align*}
                R \quad & \defeq \quad (a \, x \, | \, (y)\bar{a} \, y \, | \, p \, y)
            \end{align*}
            This term, while similar to $P$ and $Q$, reduces to the term $p \, y$, as our scoped $y$ that is fused is not the same as the free $y$ that is `printed'.
            This can be seen through $\alpha$-equivalence and renaming $(y)\bar{a} \, y$ to the equivalent term $(z)\bar{a} \, z$ and observing that $a$ is a fusion on $x$ and $z$.

            The distinction between the two behaviours is not intuitive, especially in non-trivial cases where agents may be deeply nested.
        \end{example*}

        The above example is further amplified when manipulating expressions symbolically.
        When searching for reducible patterns, the expression is divided and recombined, rearranging the expression tree and subsequently each node's parents.
        While the problem is solvable through updating which node has which parent as the expression is rearranged and manipulated, the solution eventually becomes an attempt to form a normal form out of the expression, leading to the revised method described in~\ref{sssec:calculus-analysis}.
