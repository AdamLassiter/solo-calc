\subsection{Solos Calculus}
    
    Developed by Cosimo Laneve and Bj{\"o}rn Victor in the early 2000s, the Solos calculus aims to be an improvement of the Fusion calculus.
    As such, there exists an encoding of the Fusion calculus within the Solos calculus (and hence an encoding of the $\pi$-calculus).
    The name comes from the strong distinction between the components of the calculus: \textit{solos} and \textit{agents}.
    These are roughly analogous to input/output actions and a calculus syntax similar to the $\lambda$-calculus.
    Through some clever design choices, the Solos calculus is found to have some interesting properties over other process calculi.

    \begin{definition}{(Syntax)\\}
        As defined by~\cite{solo-calculus}, the Solos calculus is constructed from \textit{solos} ranged over by $\alpha, \beta \ldots$ and \textit{agents} ranged over by $P, Q \ldots$ as such:
        \begin{center}
            \begin{tabular}{ l l l }
                $\alpha \quad \defeq$   & $u\tilde{x}$          & (input) \\
                                        & $\bar{u}\tilde{x}$    & (output) \\ \\
                $P \quad \defeq$        & $0$                   & (inaction) \\
                                        & $\alpha$              & (solo) \\
                                        & $Q \, | \, R$         & (composition) \\
                                        & $(x)Q$                & (scope) \\
                                        & $[x=y]Q$              & (match) \\
            \end{tabular}
        \end{center}
    \end{definition}


    %Examples
 
    \subsubsection{Solos Diagrams}
        The Solos calculus was further developed by~\cite{solo-diagrams} to provide a one-to-one correspondence between these  expressions and `diagram-like' objects.
        This provides a strong analog to real-world systems and an applicability to be used as a modelling tool for groups of communicating systems.

    \begin{remarks} 
        For further reading,~\cite{acyclic-solos} presents in great detail the topics of the $\pi$ and Solos calculus, Solo diagrams and furthermore Differential Interaction Nets.
    \end{remarks}


